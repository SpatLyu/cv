%% start of file `template.tex'.
%% Copyright 2006-2015 Xavier Danaux (xdanaux@gmail.com).
%
% Adapted to be an Rmarkdown template by Mitchell O'Hara-Wild
% 8 February 2019
%
% This work may be distributed and/or modified under the
% conditions of the LaTeX Project Public License version 1.3c,
% available at http://www.latex-project.org/lppl/.


\documentclass[11pt,a4paper,]{moderncv}

% moderncv themes
\moderncvstyle{casual}                             % style options are 'casual' (default), 'classic', 'banking', 'oldstyle' and 'fancy'

\definecolor{color0}{rgb}{0,0,0}% black
\definecolor{color1}{HTML}{3873B3}% custom
\definecolor{color2}{rgb}{0.45,0.45,0.45}% dark grey

\usepackage[scaled=0.86]{DejaVuSansMono}

\providecommand{\tightlist}{%
	\setlength{\itemsep}{0pt}\setlength{\parskip}{0pt}}
\def\donothing#1{#1}
\def\emaillink#1{#1}

%\nopagenumbers{}                                  % uncomment to suppress automatic page numbering for CVs longer than one page

% character encoding
%\usepackage[utf8]{inputenc}                       % if you are not using xelatex ou lualatex, replace by the encoding you are using
%\usepackage{CJKutf8}                              % if you need to use CJK to typeset your resume in Chinese, Japanese or Korean

% adjust the page margins
\usepackage[scale=0.75,footskip=60pt]{geometry}
%\setlength{\hintscolumnwidth}{3cm}                % if you want to change the width of the column with the dates
%\setlength{\makecvheadnamewidth}{10cm}            % for the 'classic' style, if you want to force the width allocated to your name and avoid line breaks. be careful though, the length is normally calculated to avoid any overlap with your personal info; use this at your own typographical risks...



% personal data
\name{}{Wenbo Lv}
\title{undergraduate}
\address{Xi'an, Shaanxi}{}{}

 % Phone number
\email{\donothing{\href{mailto:lyu.geosocial@gmail.com}{\nolinkurl{lyu.geosocial@gmail.com}}}}
\homepage{spatlyu.github.io} % Personal website


\social[github]{SpatLyu}
\social[orcid]{0009-0002-6003-3800}



% \extrainfo{additional information}                 % optional, remove / comment the line if not wanted




% Pandoc CSL macros
% definitions for citeproc citations
\NewDocumentCommand\citeproctext{}{}
\NewDocumentCommand\citeproc{mm}{%
  \begingroup\def\citeproctext{#2}\cite{#1}\endgroup}
\makeatletter
 % allow citations to break across lines
 \let\@cite@ofmt\@firstofone
 % avoid brackets around text for \cite:
 \def\@biblabel#1{}
 \def\@cite#1#2{{#1\if@tempswa , #2\fi}}
\makeatother
\newlength{\cslhangindent}
\setlength{\cslhangindent}{1.5em}
\newlength{\csllabelwidth}
\setlength{\csllabelwidth}{3em}
\newenvironment{CSLReferences}[2] % #1 hanging-indent, #2 entry-spacing
 {\begin{list}{}{%
  \setlength{\itemindent}{0pt}
  \setlength{\leftmargin}{0pt}
  \setlength{\parsep}{0pt}
  % turn on hanging indent if param 1 is 1
  \ifodd #1
   \setlength{\leftmargin}{\cslhangindent}
   \setlength{\itemindent}{-1\cslhangindent}
  \fi
  % set entry spacing
  \setlength{\itemsep}{#2\baselineskip}}}
 {\end{list}}

\usepackage{calc}
\newcommand{\CSLBlock}[1]{\hfill\break\parbox[t]{\linewidth}{\strut\ignorespaces#1\strut}}
\newcommand{\CSLLeftMargin}[1]{\parbox[t]{\csllabelwidth}{\strut#1\strut}}
\newcommand{\CSLRightInline}[1]{\parbox[t]{\linewidth - \csllabelwidth}{\strut#1\strut}}
\newcommand{\CSLIndent}[1]{\hspace{\cslhangindent}#1}

%----------------------------------------------------------------------------------
%            content
%----------------------------------------------------------------------------------
\begin{document}
%\begin{CJK*}{UTF8}{gbsn}                          % to typeset your resume in Chinese using CJK
%-----       resume       ---------------------------------------------------------
\makecvtitle



\section{Some stuff about me}\label{some-stuff-about-me}

\begin{itemize}
\tightlist
\item
  My research interests lie in \textbf{advancing methodologies in
  spatial causal inference} and \textbf{developing high-performance
  computational tools}, with a primary focus on \emph{R packages}.
\item
  Currently, my work centers on \textbf{Empirical Dynamic Modeling
  (EDM)} framework for modeling \emph{dynamic system} and
  \textbf{Difference-in-Differences (DID)} methods for \emph{event
  studies}. I am particularly interested in leveraging these approaches
  to address critical challenges in \emph{urban sustainability},
  \emph{climate change mitigation}, and broader global issues.
\item
  I possess expertise in \emph{data analysis}, \emph{statistical
  modeling}, and the development of \emph{R packages} and open-source
  analytical tools utilizing \textbf{R}, \textbf{C++}, and
  \textbf{Python}.
\item
  I have contributed to the \textbf{development} and
  \textbf{maintenance} of several open-source spatial analysis tools
  within the R community and remain dedicated to \textbf{advancing
  open-source geospatial analysis software}.
\end{itemize}

\section{Education}\label{education}

\nopagebreak
    \cventry{2021-2025}{BSc In Geographic Information Science}{Shaanxi Normal University}{Xi'an, Shaanxi}{}{\empty}

\section{Publications}\label{publications}

\phantomsection\label{refs-0ec9cd99dd6c888b81e78fc0c10b8c98}
\begin{CSLReferences}{0}{0}
\bibitem[\citeproctext]{ref-lyu2025gdverse}
\CSLLeftMargin{1. }%
\CSLRightInline{Lv, W., Lei, Y., Liu, F., Yan, J., Song, Y., \& Zhao, W.
(2025). Gdverse: An r package for spatial stratified heterogeneity
family. \emph{Transactions in GIS}, \emph{29}(2), 29:e70032.
\url{https://doi.org/10.1111/tgis.70032}}

\bibitem[\citeproctext]{ref-lyu2024veg}
\CSLLeftMargin{2. }%
\CSLRightInline{Lv, W., Liu, F., Cai, K., Cao, Y., Deng, M., Liang, W.,
Yan, J., \& Wang, G. (2024). Distinguishing the impacts and gradient
effects of climate change and human activities on vegetation cover in
the weihe river basin, china. \emph{Journal of Geophysical Research:
Biogeosciences}, \emph{129}(10).
\url{https://doi.org/10.1029/2024jg008297}}

\bibitem[\citeproctext]{ref-chen2025gcrf}
\CSLLeftMargin{3. }%
\CSLRightInline{Chen, C., Song, Y., Lv, W., Shemery, A., Hampson, K.,
Yi, W., Zhong, Y., \& Wu, P. (2025). Predicting pavement cracking
performance using laser scanning and geocomplexity‐enhanced machine
learning. \emph{Computer-Aided Civil and Infrastructure Engineering}.
\url{https://doi.org/10.1111/mice.13489}}

\bibitem[\citeproctext]{ref-song2023cuafr}
\CSLLeftMargin{4. }%
\CSLRightInline{Song, Z., Liu, F., Lv, W., \& Yan, J. (2023).
Classification of urban agricultural functional regions and their carbon
effects at the county level in the pearl river delta, china.
\emph{Agriculture}, \emph{13}(9).
\url{https://doi.org/10.3390/agriculture13091734}}

\bibitem[\citeproctext]{ref-song2023cupscos}
\CSLLeftMargin{5. }%
\CSLRightInline{Song, Z., Liu, F., \& Lv, W. (2023).
\emph{Spatiotemporal characteristics and optimization strategies of
urban-rural development disparities in china's urban agglomerations(in
chinese)} (pp. 1418--1429). People's Cities, Empowered by Planning -
Proceedings of the 2023 China Urban Planning Annual Conference (14
Regional Planning; Urban Economy).
\url{https://link.cnki.net/doi/10.26914/c.cnkihy.2023.061565}}

\end{CSLReferences}

\section{Honor}\label{honor}

\nopagebreak
    \cventry{2024.12}{Longi Non-Education Major Scholarship}{}{}{}{\empty}
    \cventry{2024.11}{First Prize in the 13th National University Student GIS Application Skills Competition}{}{}{}{\empty}
    \cventry{2024.06}{National University Student Innovation and Entrepreneurship Training Program Qualified Completion}{}{}{}{\empty}
    \cventry{2023.12}{Grand Prize in the 12th National University Student GIS Application Skills Competition}{}{}{}{\empty}
    \cventry{2023.11}{First Prize in the Second National University Student Ecological Environment Management Research Innovation Competition}{}{}{}{\empty}
    \cventry{2023.12}{Second Prize of the 5th 'Guodi Cup' National College Student Natural Resource Science and Technology Competition, China Society of Natural Resources}{}{}{}{\empty}
    \cventry{2021.10}{Outstanding Individual in Military Training Publicity for College Students, Shaanxi Normal University}{}{}{}{\empty}

\section{Unpublished}\label{unpublished}

\nopagebreak
    \cventry{First Author}{Measuring causal associations by geographical cross mapping cardinality}{Submitted to IJGIS}{}{}{\empty}
    \cventry{Third Author}{Agricultural policies reshape cropland patterns with varying impacts - a case of soybeans from Heilongjiang Province}{Submitted to Land Use Policy, currently under review}{}{}{\empty}
    \cventry{First Author}{Decomposing spatial causality through mutual information}{Plan}{}{}{\empty}
    \cventry{First Author}{On the role of explicit spatial information in stratified heterogeneity}{Plan}{}{}{\empty}
    \cventry{First Author}{Geocomplexity Mitigates Spatial Bias}{Plan}{}{}{\empty}

\section{Developed Spatial Analysis
Toolkit}\label{developed-spatial-analysis-toolkit}

\begin{itemize}
\item
  \textbf{\href{https://github.com/stscl/spEDM}{spEDM}}\\
  Inferring causation from spatial cross-sectional data through
  empirical dynamic modeling (EDM), with methodological extensions
  including geographical convergent cross mapping, geographical cross
  mapping cardinality and spatially convergent partial cross mapping, as
  well as the spatial granger causality. Data I/O is handled at the
  \textbf{R} level, while the rest is fully implemented using
  \textbf{modern C++}.
\item
  \textbf{\href{https://github.com/stscl/gdverse}{gdverse}}\\
  Detecting spatial associations via spatial stratified heterogeneity,
  accounting for spatial dependencies, interpretability, complex
  interactions, and robust stratification. In addition, it supports the
  spatial stratified heterogeneity family described in Lv et
  al.~(2025)\url{doi:10.1111/tgis.70032}. Developed using \textbf{R},
  \textbf{C++}, and \textbf{Python}.
\item
  \textbf{\href{https://github.com/stscl/sesp}{sesp}}\\
  Implements the \emph{Spatially Explicit Stratified Power} model, a
  robust framework for spatial analysis combining stratification and
  statistical power. Written in \textbf{R} and \textbf{C++}.
\item
  \textbf{\href{https://github.com/ausgis/geocomplexity}{geocomplexity}}\\
  Focuses on mitigating spatial biases by leveraging geographical
  complexity. Combines computational efficiency with flexibility,
  developed in \textbf{C++}, \textbf{R}, and \textbf{C}.
\item
  \textbf{\href{https://github.com/stscl/cisp}{cisp}}\\
  Introduces a novel \emph{Correlation Indicator Based on Spatial
  Patterns} for measuring spatial correlations with high precision.
  Written in \textbf{R}.
\item
  \textbf{\href{https://github.com/ausgis/geosimilarity}{geosimilarity}}\\
  Provides methods for calculating \emph{Geographically Optimal
  Similarity}, enabling better spatial predict. Developed in \textbf{R}.
\item
  \textbf{\href{https://github.com/ausgis/GD}{GD}}\\
  Implements \emph{Geographical Detectors}, a toolkit for assessing
  spatial factors influencing heterogeneity. Fully developed in
  \textbf{R}.
\item
  \textbf{\href{https://github.com/stscl/sdsfun}{sdsfun}}\\
  Adds complementary features for \emph{Spatial Data Science}, providing
  user-friendly functionalities for geospatial research. Developed using
  \textbf{C++} and \textbf{R}.
\item
  \textbf{\href{https://github.com/stscl/geocn}{geocn}}\\
  Simplifies the process of loading and managing spatial datasets of
  China, supporting research with localized datasets. Developed in
  \textbf{R}.
\item
  \textbf{\href{https://github.com/r-spatial/qgisprocess}{qgisprocess}}\\
  Offers an \textbf{R} interface to \emph{QGIS processing algorithms},
  enabling seamless integration of QGIS functionalities into R
  workflows.
\item
  \textbf{\href{https://github.com/SpatLyu/spEcula}{spEcula}}\\
  Provides advanced methods for \emph{Spatial Prediction} in \textbf{R},
  supporting applications ranging from environmental modeling to spatial
  econometrics.
\item
  \textbf{\href{https://github.com/SpatLyu/tidyrgeoda}{tidyrgeoda}}\\
  Offers a tidy interface for \emph{rgeoda}, bridging geospatial
  analysis and tidyverse workflows to streamline data handling and
  modeling in \textbf{R}.
\end{itemize}


\end{document}

%\clearpage\end{CJK*}                              % if you are typesetting your resume in Chinese using CJK; the \clearpage is required for fancyhdr to work correctly with CJK, though it kills the page numbering by making \lastpage undefined
\end{document}


%% end of file `template.tex'.
